\chapter{Conclusion}

%The inclusive jet and dijet cross sections have been measured using 37$\mathrm{pb}^{-1}$ of data collected by \atlas during 2010. This is a precise measurement of QCD in a region of phase space that had not previously been explored. 
%
%Test beam measured. Simulation validated wrt data.
%
%
%
%Analysed 2003 beam test data. Software bug affecting data reconstruction was fixed, results presented here supercede those that were previously published. effect of material on response and resolution has been measured. Data also compared to MC, results generally agree well. Good agreement between data and simulation found when studying the cluster moments used for LHC.
%
%Inclusive jet and dijet cross sections have been measured using 37 of data collected with \atlas during 2010. While the CMS measurements contain a gap in the rapidity coverage, the cross section measurement presented here fully covers the rapidity region from 0 to 4.4. This was accomplished through the use of a sophisticated trigger scheme, which allowed jets in the transition region to be accepted by either the central or the forward jet trigger system. This analysis constitutes a precise measurement of QCD in a region of phase space that had not previously been explored. 

Data taken during the 2003 beam test of the \atlas forward calorimeter has been analysed. A software bug affecting data reconstruction was fixed, and thus the results presented here supercede those that were previously published. The effects of additional upstream material on the calorimeter response and resolution have been studied. The results of Monte Carlo simulations of the beam test have been compared to the results obtained from data, and the agreement between the two is typically good. Good agreement between data and simulation is also found when studying the topological cluster moments used in local hadronic calibration, and these results are consistent with those obtained from other studies. These studies play a key role in the validation of Monte Carlo simulations for the \atlas forward calorimeter, which is important because all the hadronic calibration schemes used by \atlas are derived from Monte Carlo studies.


The inclusive jet and dijet cross sections have been measured using 37$\mathrm{pb}^{-1}$ of data collected with the \atlas detector during 2010. While the CMS measurements from the same time period contain a gap in the rapidity coverage, the measurements presented here fully cover the rapidity region $0 < |y| < 4.4$. This was accomplished through the use of a sophisticated trigger scheme, which allowed jets in the transition region to be accepted by either the central or the forward jet trigger system. Theoretical predictions generally agree well with the measured cross sections, with the \powheg generator giving results that agree well even at high \pt~and rapidity. These measurements constitute a new measurement of QCD in a region of phase space that had not previously been explored, spanning two orders of magnitude in jet \pt~(dijet mass) and eleven orders of magnitude in cross section.

%suckit
