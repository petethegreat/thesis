\chapter{Introduction}

This thesis will describe the measurement of the inclusive jet and dijet cross section using 2010 data obtained from ATLAS detector, as well as data and simulation results related to the 2003 beam test of the ATLAS forward calorimeter.
Chapter~\ref{chap_SM} discusses the Standard Model of particle physics, focusing on aspects of QCD that are relevent for jet production and Monte Carlo event generators. Chapter~\ref{chap_detector} covers the LHC and the \atlas detector. As the calorimeters are used for jet reconstruction, these are discussed in detail. Particular attention is paid to the forward calorimeter, as it is also the subject of chapter~\ref{chapTB}, which describes the 2003 beam test. This chapter details measurements of the intrinsic response and resolution of the calorimeter, as well as measuring the effects of additional material placed upstream of the calorimeter. Monte Carlo simulations of the beam test have also been carried out, and the results of these simulations are compared to data. The testbeam offers a clean environment in which the FCal can be calibrated and where data can be compared to Monte Carlo. As the tracking coverage at \atlas does not cover the region occupied by the forward calorimeter, direct comparisons of data and Monte Carlo using particles of known energy are difficult to perform in-situ. It is particularly important to validate the simulation in this environment, as all of the hadronic calibration schemes used at \atlas are derived from Monte Carlo. Calibration factors derived from these testbeam data form the basis of the forward calorimeter calibration used by \atlas. The forward calorimeter also plays an important role in the measurement of the inclusive jet and dijet cross sections. These measurements are discussed in chapter~\ref{INCJETCHAPTER}, and are made using data collected with \atlas during 2010. The analysis presented here expands upon earlier measurements, as data from the forward calorimeter is included in order to enlarge the kinematic reach of the measurements. 

%are used in chapter~\ref{INCJETCHAPTER}, which describes the measurement of the inclusive jet and dijet cross sections using data collected with \atlas during 2010. This analysis expands upon earlier measurements, as data from the forward calorimeter is included in order to enlarge the kinematic reach of the measurement. 



%response
%effect of upstream material
%data/MC
% 
%
%
% Chapter~\ref{chap_detector} discusses the LHC and the ATLAS detector, while chapter \ref{chap_SM} discusses the Standard Model, in particular QCD.
%The measurement of the inclusive jet and dijet cross sections are discussed in chapter \ref{INCJETCHAPTER}.
%Chapters \ref{chap_TB_intro} and \ref{chap_TB_results} focus on the 2003 test beam. An overview of the beam test environment and a discussion of the simulation is provided in chapter \ref{chap_TB_intro}, while chapter \ref{chap_TB_results} discusses the test beam results and compares these to the results of the simulation.
%
%%that's my introduction, bitches.

%The other thing I think you could do today is to do a little more work on the introduction. I think it is
%important to more clearly lay out here what is presented in the thesis and to try to tie together the
%two topics (which do fit, in a way, since the jet measurements rely on FCal energy measurements).
%In the end, it's your thesis, so it's up to you to decide how to write this, but let me point out that, at the 
%moment, the first place that you make the point that all the ATLAS hadronic calibration schemes are 
%based on Monte Carlo, is on page 176 in the (rather brief) conclusions. I think you should make this 
%point already in this introduction. You could (if you wish) make a general statement about how all 
%calorimeters rely in part on testbeam results for their calibrations (and when you discuss the electronic 
%calibration later on, you could point to the factors that come from testbeam studies) but you can also 
%make the point that since there is no tracking in front of the FCal, in-situ studies of the FCal are difficult, 
%so the testbeam data and Monte Carlo comparisons are particularly important. 
